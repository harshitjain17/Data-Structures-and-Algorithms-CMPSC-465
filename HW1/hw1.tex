\documentclass[letterpaper,11pt]{article}

\usepackage{geometry}
\usepackage{pslatex}
\usepackage{fancyhdr}
\usepackage{graphicx}
\usepackage{color}
\usepackage{enumitem}
\usepackage{amssymb}
\geometry{ margin = 1.0in }

%%% TODO modify these variables %%%
\def\homeworknum{1}
\def\myname{Harshit Jain}
\def\myaccessid{hmj5262}
\def\myrecitation{8}
%%%%

\pagestyle{fancy}
\lhead{{\bf CMPSC 465 Fall 2022}}
\chead{{\bf Assignment~\homeworknum}}
\rhead{{\bf \today}}

\newcounter{problemid}
%\stepcounter{problemid}
\def\newproblem{\clearpage\newpage{\bf Problem~\arabic{problemid}\stepcounter{problemid}}\hfill\fbox{\parbox{0.16\textwidth}{\bf Points:}}\par}

\setlength\parindent{0em}
\setlength\parskip{8pt}
\setlength{\fboxsep}{6pt}


\begin{document}

\framebox[\textwidth]{
	\parbox{0.96\textwidth}{
		\parbox{0.12\textwidth}{\bf Name:}\parbox{0.6\textwidth}{\myname}\\
		\parbox{0.12\textwidth}{\bf Access ID:}\parbox{0.6\textwidth}{\myaccessid}\\
		\parbox{0.12\textwidth}{\bf Recitation:}\parbox{0.6\textwidth}{\myrecitation}
	}
}


%% your solutions %%%

\newproblem
\textbf{Acknowledgements}
\begin{enumerate}[label=(\alph*)]
    \item I did not work in a group.
    \item I did not consult without anyone my group members.
    \item I did not consult any non-class materials.
\end{enumerate}


% PROBEM 1
\newproblem
\textbf{Compare Growth Rates}
\begin{enumerate}[label=(\alph*)]
    \item $ n^{1.5} = \Omega (n^{1.3}) \; ~ \because n^{1.3} < n^{1.5} $
    \item $ 2^{n-1} = \Theta (2^{n}) \; ~ \because 2^{n-1} = \frac{2^{n}}{2} $ and constants (in this case, $\frac{1}{2}$) does not matter
    \item $ n^{1.3logn} = \Omega (n^{1.5}) \; ~ \because$ asymptotically $n^{1.3logn} > n^{1.5}$
    \item $ 3^{n} = \Omega (n \cdot 2^{n}) \; ~ \because$ asymptotically $3^{n} > n \cdot 2^{n}$
    \item $ (logn)^{100} = O (n^{0.1}) \; ~ \because $ power of $ logn$ will only weaken the infinity of $logn$ and $n^{0.1}>(logn)^{100}$
    \item $ n = \Omega ((logn)^{log(logn)}) \; ~ \because $ infinity of $(logn)^{log(logn)} > $ infinity of $n$
    \item $ 2^{n} = \Omega (n!) \; ~ \because $ infinity of exponential function is greater than the infinity of factorial function  
    \item $ log(e^{n}) = O (n \cdot logn) \; ~ \because log(e^{n}) = n $ and we know that $n < nlogn$
    \item $ n + logn = \Theta (n + (logn)^2) \; ~ \because $ dominating term is $n$ and powers of $logn$ will not affect infinities
    \item $ 5n + \sqrt{n} = \Omega (logn + n) \; ~ \because $ comparable functions are $\sqrt{n}$ and $logn$; $\sqrt{n} > logn$
\end{enumerate}

% PROBLEM 2
\newproblem


\newproblem
Your solution starts here ...


\end{document}
