\documentclass[letterpaper,11pt]{article}

\usepackage{geometry}
\usepackage{pslatex}
\usepackage{fancyhdr}
\usepackage{graphicx}
\usepackage{color}
\usepackage{enumitem} % for ordered list labels
\usepackage{amssymb} % for symbols
\usepackage{scrextend} % for indentation
\usepackage{tabto} % for tabs
\usepackage{amsmath} % for text in equation
\usepackage{listings} % to highlight code

% Define a custom color
\definecolor{backcolour}{rgb}{0.95,0.95,0.92}
\definecolor{codegreen}{rgb}{0,0.6,0}

% Define a custom style
\lstdefinestyle{myStyle}{
    backgroundcolor=\color{backcolour},   
    commentstyle=\color{codegreen},
    basicstyle=\ttfamily\footnotesize,
    breakatwhitespace=false,         
    breaklines=true,                 
    keepspaces=true,                 
    numbers=left,       
    numbersep=5pt,                  
    showspaces=false,                
    showstringspaces=false,
    showtabs=false,                  
    tabsize=2,
}

% Use \lstset to make myStyle the global default
\lstset{style=myStyle}

\graphicspath{ {./} }
\geometry{ margin = 1.0in }

%%% TODO modify these variables %%%
\def\homeworknum{4}
\def\myname{Harshit Jain}
\def\myaccessid{hmj5262}
\def\myrecitation{8}
%%%%

\pagestyle{fancy}
\lhead{{\bf CMPSC 465 Fall 2022}}
\chead{{\bf Assignment~\homeworknum}}
\rhead{{\bf \today}}

\newcounter{problemid}
%\stepcounter{problemid}
\def\newproblem{\clearpage\newpage{\bf Problem~\arabic{problemid}\stepcounter{problemid}}\hfill\fbox{\parbox{0.16\textwidth}{\bf Points:}}\par}

\setlength\parindent{0em}
\setlength\parskip{8pt}
\setlength{\fboxsep}{6pt}


\begin{document}

\framebox[\textwidth]{
	\parbox{0.96\textwidth}{
		\parbox{0.12\textwidth}{\bf Name:}\parbox{0.6\textwidth}{\myname}\\
		\parbox{0.12\textwidth}{\bf Access ID:}\parbox{0.6\textwidth}{\myaccessid}\\
		\parbox{0.12\textwidth}{\bf Recitation:}\parbox{0.6\textwidth}{\myrecitation}
	}
}


%% your solutions %%%

\newproblem
\textbf{Acknowledgements}
\begin{enumerate}[label=(\alph*)]
    \item I did not work in a group.
    \item I did not consult without anyone my group members.
    \item I did not consult any non-class materials.
\end{enumerate}


% PROBLEM 1
\newproblem
\textbf{Divide-and-Conquer}
\begin{enumerate}[label=(\alph*)]
    
    \item We will first have the function which returns the frequency of the element in the given list. This function will take $O(n)$ time.
    
    \begin{lstlisting}[language=Python]
    def frequencyCalculator(Array, element):
        count = 0
        for ele in Array:
            if (element == ele):
                count += 1
        return count
    \end{lstlisting}

    Now, to start, we will split the array $A$ into $2$ subarrays $A_1$ and $A_2$ of half the size. Then we will calculate the majority elements of $A_1$ and $A_2$. 
    The algorithm is as follows$:$

    \begin{lstlisting}[language=Python]
    def majorityElement(Array, low, high):
    
        subArray = Array[low:high+1]
        
        # base case
        if (len(subArray)==1):
            return subArray[0]
        
        mid = (low+high)//2
        leftMajorityElement = majorityElement(Array, low, mid)
        rightMajorityElement = majorityElement(Array, mid+1, high)
        
        if (leftMajorityElement == rightMajorityElement):
            return leftMajorityElement
        
        leftFrequency = frequencyCalculator(subArray, leftMajorityElement)
        rightFrequency = frequencyCalculator(subArray, rightMajorityElement)
        
        if (leftFrequency > len(subArray)//2):
            return leftMajorityElement
        elif (rightFrequency > len(subArray)//2):
            return rightMajorityElement
        else:
            return -1 # no majority element found
    \end{lstlisting}
    
    \item 

    \item Here, we are choosing the majority element of sub-arrays after dividing them in half.
    Moreover, frequency calculation is using $O(n)$ time as showen above.
    Therefore, the recurrence relation for this algorithm is given by$:$ \[T(n)=2T(n/2)+O(n)\]
    According to Master's Theorem, the time complexity$: \underline{O(n logn)}$

\end{enumerate}



% PROBLEM 2
\newproblem
\textbf{Reverse graph}
\begin{enumerate}[label=(\alph*)]
    
    \item 

    \item 

\end{enumerate}


% PROBLEM 3
\newproblem
\textbf{Graph Basics}
\begin{enumerate}[label=(\alph*)]
    
    \item 
    \begin{minipage}{.22 \linewidth}
		Adjacency-list

		\begin{tabular}{l | l}

		$A$ & $\rightarrow B \rightarrow E$\\
		$B$ & $\rightarrow G \rightarrow D$\\
		$C$ & $\rightarrow I \rightarrow H$\\
		$D$ & $\rightarrow E$\\
		$E$ & $\rightarrow D$\\
		$F$ & $\rightarrow$\\
		$G$ & $\rightarrow D \rightarrow F$\\
		$H$ & $\rightarrow I$\\
		$I$ & $\rightarrow$ \\
		\end{tabular}\\ 
	\end{minipage}

    \item The most number of edges that an undirected graph can have are $: \frac{|V|(|V|-1)}{2}$

    \item 

    \item 

    \item 
\end{enumerate}
\end{document} 
