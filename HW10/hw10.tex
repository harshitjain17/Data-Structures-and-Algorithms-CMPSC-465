\documentclass[letterpaper,11pt]{article}

\usepackage{geometry}
\usepackage{pslatex}
\usepackage{fancyhdr}
\usepackage{graphicx}
\usepackage{color}
\usepackage{enumitem} % for ordered list labels
\usepackage{amssymb} % for symbols
\usepackage{scrextend} % for indentation
\usepackage{tabto} % for tabs
\usepackage{amsmath} % for text in equation
\usepackage{forest, tikz} % to make forest
\usepackage{mathtools}
\usepackage[ruled, lined, linesnumbered, commentsnumbered, longend]{algorithm2e}

\graphicspath{ {./} }
\geometry{ margin = 1.0in }

%%% TODO modify these variables %%%
\def\homeworknum{10}
\def\myname{Harshit Jain}
\def\myaccessid{hmj5262}
\def\myrecitation{8}
\def\lc{\left\lceil}   
\def\rc{\right\rceil}
%%%%

\pagestyle{fancy}
\lhead{{\bf CMPSC 465 Fall 2022}}
\chead{{\bf Assignment~\homeworknum}}
\rhead{{\bf \today}}

\newcounter{problemid}
%\stepcounter{problemid}
\def\newproblem{\clearpage\newpage{\bf Problem~\arabic{problemid}\stepcounter{problemid}}\hfill\fbox{\parbox{0.16\textwidth}{\bf Points:}}\par}

\setlength\parindent{0em}
\setlength\parskip{8pt}
\setlength{\fboxsep}{6pt}


\begin{document}

\framebox[\textwidth]{
	\parbox{0.96\textwidth}{
		\parbox{0.12\textwidth}{\bf Name:}\parbox{0.6\textwidth}{\myname}\\
		\parbox{0.12\textwidth}{\bf Access ID:}\parbox{0.6\textwidth}{\myaccessid}\\
		\parbox{0.12\textwidth}{\bf Recitation:}\parbox{0.6\textwidth}{\myrecitation}
	}
}


%% your solutions %%%

\newproblem
\textbf{Acknowledgements}
\begin{enumerate}[label=(\alph*)]
    \item I did not work in a group.
    \item I did not consult with anyone in my group members.
    \item I did not consult any non-class materials.
\end{enumerate}


% PROBLEM 1
\newproblem
\begin{algorithm}
    
    \caption{GREEDY-HORN}

    \SetKwInOut{KwIn}{Input}
    \SetKwInOut{KwOut}{Output}

    \KwIn{set of Horn clauses}
    \KwOut{either the assignment or "unsatisfiable"}
    
    Set all variables to $0$;

    \While{$\exists$ an "$\implies$" that is not satisfied}{
        Set its RHS to $1$;
    }
    
    \If{all pure negative clauses are $1$}{
        \Return{ the assignment}
    }
    
    \Else{
        \Return{"unsatisfiable"}
    }

\end{algorithm}

\begin{enumerate}[label=(\alph*)]
    
    \item 
    According to the algorithm, first set all variables to $0 \implies$ \[w=0, x=0, y=0, z=0\]
    
    Now there are $5$ clauses having "$\implies$" out of which $4^{th}$ clause ($\implies x$) is not satified. So, we will set RHS of this clause to $1$, that is, $\textbf{x=1}$.

    This will lead to reconsidering the assignment of $y$ because according to $3^{rd}$ clause ($x \implies y$), if LHS is True then RHS should be set to $1$, that is, $\textbf{y=1}$.

    This will lead to reconsidering the assignment of $w$ because according to $5^{th}$ clause ($x \land y \implies w$), if LHS is True then RHS should be set to $1$, that is, $\textbf{w=1}$.
    
    This will lead to reconsidering the assignment of $z$ because according to $1^{st}$ clause ($w \land y \land z \implies x$), if RHS is True then LHS should be resolved to $1$, that is, $\textbf{z=1}$.
    
    Now, pure negative clauses are failed to satisfy, so there is no satisfying assignment, hence algorithm will return \textbf{"unsatisfiable"}.
    
    \item
    According to the algorithm, first set all variables to $0 \implies$ \[w=0, x=0, y=0, z=0\]

    Now there are $4$ clauses having "$\implies$" out of which $4^{th}$ clause ($\implies z$) is not satified. So, we will set RHS of this clause to $1$, that is, $\textbf{z=1}$.

    This will lead to reconsidering the assignment of $w$ because according to $2^{nd}$ clause ($z \implies w$), if LHS is True then RHS should be set to $1$, that is, $\textbf{w=1}$.

    Here, $x$ and $y$ need not to be changed since the implications are still satified with having $\textbf{x=0, y=0}$.

    Now, pure negative clauses are still $1$ and hence, satisfied. So, the algorithm will return the assigment \[\textbf{w=1, x=0, y=0, z=1}\]

\end{enumerate}


% PROBLEM 2
\newproblem


% PROBLEM 3
\newproblem
\underline{Subproblem $\rightarrow$ Notation:} LCS$(i, j) =$ length of the longest common substring for which there are indices $i$ and $j$ with $x_{i}x_{i+1} \cdots x_{i+k-1} = y_{j}y_{j+1} \cdots y_{j+k-1}$ where $1 \leq i \leq n$, $1 \leq j \leq m$.

\underline{Recurrence $\rightarrow$ Computing LCS($i$, $j$):} 
    \begin{equation}
        \text{LCS}(i, j) =
            \begin{cases}
                1 + \text{LCS}(i-1, j-1) & \text{; if } x[i] = y[j]\\
                0 & \text{; otherwise }
            \end{cases}
    \end{equation}

\underline{Base Case:} LCS$(i, 0) = 0$, LCS$(0, j) = 0$

\underline{Pseudocode:}

\begin{algorithm}
    
    \caption{Longest Common Substring}
    
    \SetKwComment{Comment}{/* }{ */}
    \SetKwInOut{KwIn}{Input}
    \SetKwInOut{KwOut}{Output}

    \KwIn{$x = x_{1}x_{2} \cdots x_{n}, y = y_{1}y_{2} \cdots y_{m}$}
    \KwOut{$k =$ length of the longest common string}
    
    Set $k = 0$;
    
    \For{i = 0 to n}{
        \For{j = 0 to m}{
            \If{i == 0 or j == 0}{
                LCS$[i][j] =  0$ \Comment*[r]{Base Case}
            }
            \ElseIf{x[i] == y[j]}{
                LCS$[i][j] = 1 +$ LCS$[i-1][j-1]$ \Comment*[r]{Recurrence Formula}
                
                \If{LCS[i][j] $\geq$ k}{
                    $k =$ LCS$[i][j]$ \Comment*[r]{Optimal Solution}
                    solutionRow $= i$ \Comment*[r]{The row holding Optimal Solution}
                    solutioncolumn $= j$ \Comment*[r]{The column holding Optimal Solution}
                }
            }
            \Else{
                LCS$[i][j] = 0$
            }
        }
    }
    \Return{$k$}

\end{algorithm}





\underline{Explanation:}

The approach is to find the length of the longest common substring for all substrings of both the strings $x$ and $y$ and store these lengths in a table LCS. Each cell $(i,j)$ of the table LCS, that is, LCS$[i][j]$ either holds $0$ if $x[i] \neq y[j]$, or holds the length of the common substring of $x[0 \cdots i]$ and $y[0 \cdots j]$ if $x[i] = y[j]$, which is made up including $x[i]$ and $y[j]$. In that way, the table keeps track of all the common substrings available and returns the largest length of common substring available in the table.

If the character happened to be the same while iterating through both the strings $x$ (interator $i$) and $y$ (interator $j$), then this algorithm checks for the similarity of until previous character of both the strings ($x[i-1] == y[j-1]$) which has already been stored in the table, and add $1$ to it which signifies $x[i] = y[j]$.






\end{document} 