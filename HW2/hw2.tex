\documentclass[letterpaper,11pt]{article}

\usepackage{geometry}
\usepackage{pslatex}
\usepackage{fancyhdr}
\usepackage{graphicx}
\usepackage{color}
\usepackage{enumitem} % for ordered list labels
\usepackage{amssymb} % for symbols
\usepackage{scrextend} % for indentation
\usepackage{tabto} % for tabs
\usepackage{amsmath} % for text in equation

\graphicspath{ {./} }
\geometry{ margin = 1.0in }

%%% TODO modify these variables %%%
\def\homeworknum{2}
\def\myname{Harshit Jain}
\def\myaccessid{hmj5262}
\def\myrecitation{8}
%%%%

\pagestyle{fancy}
\lhead{{\bf CMPSC 465 Fall 2022}}
\chead{{\bf Assignment~\homeworknum}}
\rhead{{\bf \today}}

\newcounter{problemid}
%\stepcounter{problemid}
\def\newproblem{\clearpage\newpage{\bf Problem~\arabic{problemid}\stepcounter{problemid}}\hfill\fbox{\parbox{0.16\textwidth}{\bf Points:}}\par}

\setlength\parindent{0em}
\setlength\parskip{8pt}
\setlength{\fboxsep}{6pt}


\begin{document}

\framebox[\textwidth]{
	\parbox{0.96\textwidth}{
		\parbox{0.12\textwidth}{\bf Name:}\parbox{0.6\textwidth}{\myname}\\
		\parbox{0.12\textwidth}{\bf Access ID:}\parbox{0.6\textwidth}{\myaccessid}\\
		\parbox{0.12\textwidth}{\bf Recitation:}\parbox{0.6\textwidth}{\myrecitation}
	}
}


%% your solutions %%%

\newproblem
\textbf{Acknowledgements}
\begin{enumerate}[label=(\alph*)]
    \item I did not work in a group.
    \item I did not consult without anyone my group members.
    \item I did not consult any non-class materials.
\end{enumerate}


% PROBLEM 1
\newproblem
\textbf{Analyze Running Time}
\begin{enumerate}
    \item \textbf{Runtime: $\theta (n^2)$}
    
    When $i=1$, then $j=1$. Here, while loop runs for $\frac{n-1}{5}$ times.
    
    When $i=2$, then $j=2$. Here, while loop runs for $\frac{n-2}{5}$ times.
    
    \tabto{40pt} $\vdots$

    When $i=(n-5)$, then $j=(n-5)$. Here, while loop runs for $1$ time.

    \tabto{40pt} $\vdots$

    When $i=(n-1)$, then $j=(n-1)$. Here, while loop runs for $1$ time.

    Total time: $\frac{n-1}{5} + \frac{n-2}{5} + \dots + \frac{5}{5} + 1 + 1 + 1 + 1 + 1 \Rightarrow \theta (n^2)$
    
    \item \textbf{Runtime: $\theta (n^2)$}
    
    When $i=1$, then while loop runs from $4$ to $n$. Time $=(n-4)$
    
    When $i=2$, then while loop runs from $8$ to $n$. Time $=(n-8)$
    
    \tabto{40pt} $\vdots$

    When $i=\frac{n}{4}$, then while loop will run $1$ time. Time $=1$

    For loop will run for total of $n$ times.

    $\Rightarrow \theta (n^2)$

    \item \textbf{Runtime: $\theta (n)$}
    
    The loop will run for $\frac{n}{2} + \frac{n}{2^2} + \frac{n}{2^3} + \dots = n (\frac{1}{2} + \frac{1}{2^2} + \frac{1}{2^3} + \dots) = \theta (n)$

    So, total time $\Rightarrow \theta (n)$

\end{enumerate}



% PROBLEM 2
\newproblem
\textbf{Polynomials and Horner’s rule}
\[ P(x) = a_0 + a_1 x^1 + a_2 x^2 + \cdots + a_n x^n; x = x_0\]
\begin{enumerate}[label=(\alph*)]
    \item \underline{For Addition:} The Brute Force Algorithm will perform $\textbf{n}$ additions in total.
    
    So, Time taken for additions $= O(n)$

    \underline{For Multiplication:} $P(x) = a_0 + [a_1 \cdot x_0] + [a_2 \cdot x_2 \cdot x_2] + \cdots + [a_n \cdot x_n \cdot x_n \cdots (n times)]$

    So, Time taken $= T_n = 1 + 2 + 3 + \cdots + n$
    
    \tabto{40pt} $= \frac{n(n+1)}{2}$

    \tabto{40pt} $= \frac{n^2}{2} + \frac{n}{2}$

    \tabto{40pt} $= O(n^2)$

    Therefore, total time complexity for Brute Force Algorithm $= O(n^2) + O(n) = O(n^2)$

    \item \[LI = \sum_{i=0}^{n-1} (a_{n-i} \cdot x^{n-i-1})\]
    
    \textbf{Initialization:} At first iteration, the Loop Invariant (LI) holds where $z=a_0$.
    This represents the co-efficient for $P(x)$ with the maximum co-efficient of 0. Therefore, this is True. 

    \textbf{Maintenance:} At $i^{th}$ iteration, $z_i = z_{i-1} + a_i$.
    
    Assume that the Loop Invariant holds for $z_{i-1} = \sum_{k=0}^{n-i-1} (a_{k+i+1} \cdot x^k \cdot x_0)$
    
    Algebraically, $z_i = (z_{i-1} \cdot x_0) + a_i x_0^0$

    \tabto{74pt} $= \sum_{k=0}^{n-i} a_{k+1} \cdot x_0^k$

    At $i=-1$, we have: $z_{-1} = \sum_{k=0}^{n} a_k \cdot x_0^k$

    Therefore, if LI holds for $i-1$, then it holds for $i$.

    Thus, the algorithm is correct.


    \textbf{Termination:} LI holds at the start of the iteration $n$ means tht the algorithm is correct.
    \item \underline{For Addition:} This algorithm use $O(n)$ additions. 
    
    \underline{For Multiplication:} This algorithm use $2n-1 = O(n)$ multiplication.
\end{enumerate}


% PROBLEM 3
\newproblem
\textbf{Solving recurrences}
\begin{enumerate}[label=(\alph*)]
    \item Height $= \log_2 n$
    
    Branching factor $= 2$

    The size of the sub-problems at depth $k = \frac{n}{2^k}$

    Number of sub-problems at depth $k = 2^k$

    Total work done $=$ 
    
    \[ \sum_{k=0}^{\log_2 n} (2^k) \cdot (\frac{n}{2^k})^{\frac{1}{2}}
    \Rightarrow \sqrt{n} \cdot \sum_{k=0}^{\log_2 n} (2^\frac{k}{2})
    \Rightarrow \sqrt{n} \cdot \theta (\sqrt{n})
    \Rightarrow \theta (n) \]
    
    (Note that, $\sum_{k=0}^{\log_2 n} (2^\frac{k}{2})$ is a geometric series $= 2^\frac{0}{2} + 2^\frac{1}{2} + \cdots + 2^\frac{\log_2 n}{2} = \theta (\sqrt{n})$)
    
    \item Height $= \log_3 n$
    
    Branching factor $= 2$

    The size of the sub-problems at depth $k = \frac{n}{3^k}$

    Number of sub-problems at depth $k = 2^k$

    Total work done $=$ 
    
    \[ \sum_{k=0}^{\log_3 n} (2^k) \cdot (\frac{n}{3^k})^0
    \Rightarrow \sum_{k=0}^{\log_3 n} (2^k)
    \Rightarrow 2^0 + 2^1 + 2^3 + \cdots + 2^{\log_3 n}
    \Rightarrow \theta (n^{\log_3 2}) \]
     
    \item Height $= \log_4 n$
    
    Branching factor $= 5$

    The size of the sub-problems at depth $k = \frac{n}{4^k}$

    Number of sub-problems at depth $k = 5^k$

    Total work done $=$ 
    
    \[ \sum_{k=0}^{\log_4 n} (5^k) \cdot (\frac{n}{4^k})^1
    \Rightarrow n \cdot \sum_{k=0}^{\log_4 n} ((\frac{5}{4})^k)
    \Rightarrow n \cdot ((\frac{5}{4})^0 + (\frac{5}{4})^1 + \cdots + (\frac{5}{4})^{\log_4 n})
    \Rightarrow n \cdot \theta (\frac{5^{\log_4 n}}{n})
    \Rightarrow \theta (n^{log_4 5}) \]
    
    \item Height $= \log_7 n$
    
    Branching factor $= 7$

    The size of the sub-problems at depth $k = \frac{n}{7^k}$

    Number of sub-problems at depth $k = 7^k$

    Total work done $=$ 
    
    \[ \sum_{k=0}^{\log_7 n} (7^k) \cdot (\frac{n}{7^k})^1
    \Rightarrow n \cdot \sum_{k=0}^{\log_7 n} 1
    \Rightarrow n \cdot \log n
    \Rightarrow \theta (n \log n) \]
    
    \item Height $= \log_3 n$
    
    Branching factor $= 9$

    The size of the sub-problems at depth $k = \frac{n}{3^k}$

    Number of sub-problems at depth $k = 9^k$

    Total work done $=$ 
    
    \[ \sum_{k=0}^{\log_3 n} (9^k) \cdot (\frac{n}{3^k})^2
    \Rightarrow n^2 \cdot \sum_{k=0}^{\log_3 n} 1
    \Rightarrow n^2 \cdot \log n
    \Rightarrow \theta (n^2 \log n) \]

\end{enumerate}
\end{document} 
