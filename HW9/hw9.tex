\documentclass[letterpaper,11pt]{article}

\usepackage{geometry}
\usepackage{pslatex}
\usepackage{fancyhdr}
\usepackage{graphicx}
\usepackage{color}
\usepackage{enumitem} % for ordered list labels
\usepackage{amssymb} % for symbols
\usepackage{scrextend} % for indentation
\usepackage{tabto} % for tabs
\usepackage{amsmath} % for text in equation
\usepackage{forest, tikz} % to make forest
\usepackage{mathtools}
\usepackage[ruled, lined, linesnumbered, commentsnumbered, longend]{algorithm2e}

\graphicspath{ {./} }
\geometry{ margin = 1.0in }

%%% TODO modify these variables %%%
\def\homeworknum{9}
\def\myname{Harshit Jain}
\def\myaccessid{hmj5262}
\def\myrecitation{8}
\def\lc{\left\lceil}   
\def\rc{\right\rceil}
%%%%

\pagestyle{fancy}
\lhead{{\bf CMPSC 465 Fall 2022}}
\chead{{\bf Assignment~\homeworknum}}
\rhead{{\bf \today}}

\newcounter{problemid}
%\stepcounter{problemid}
\def\newproblem{\clearpage\newpage{\bf Problem~\arabic{problemid}\stepcounter{problemid}}\hfill\fbox{\parbox{0.16\textwidth}{\bf Points:}}\par}

\setlength\parindent{0em}
\setlength\parskip{8pt}
\setlength{\fboxsep}{6pt}


\begin{document}

\framebox[\textwidth]{
	\parbox{0.96\textwidth}{
		\parbox{0.12\textwidth}{\bf Name:}\parbox{0.6\textwidth}{\myname}\\
		\parbox{0.12\textwidth}{\bf Access ID:}\parbox{0.6\textwidth}{\myaccessid}\\
		\parbox{0.12\textwidth}{\bf Recitation:}\parbox{0.6\textwidth}{\myrecitation}
	}
}


%% your solutions %%%

\newproblem
\textbf{Acknowledgements}
\begin{enumerate}[label=(\alph*)]
    \item I worked with Yug Jarodiya.
    \item I did not consult with anyone in my group members.
    \item I did not consult any non-class materials.
\end{enumerate}


% PROBLEM 1
\newproblem

\underline{Given:} A Set $C = \{0,1,...,n-1\}$ of characters

\underline{To show:} Represent any optimal prefix-free code on $C$ using only $(2n - 1 + n \lc \log n \rc)$ bits

\underline{Proof:}

$\Rightarrow$ An optimal prefix-free code on $C$ has an associated full binary tree with $n$ leaves and $(n-1)$ internal vertices. This tree can be coded by a sequence of  $n+(n-1) = (2n - 1)$ bits.

$\Rightarrow$ Height of full binary tree = $\lc \log n \rc$

$\Rightarrow$ To associate the $n$ members of $C\{0,1,...,n-1\}$ with the $n$ leaves of the tree, which can be done by listing them in the order in which pre-order traversal encounters them, $\lc \log n\rc$ is enough bits to represent each of the $n$ members of $C$. This is because, each member in $C$ (leaf of the tree) has the maximum possible depth of $\lc \log n \rc$ in a full binary tree. For $n$ leaves, it require $n \lc \log n\rc $ bits to represent.

$\Rightarrow$ Total bits needed to represent optimal prefix-free code on $C =$
\[ n \emph{\text{ (for leaves)}} + (n - 1) \emph{\text{ (for internal nodes)}} + n \lc \log n \rc \Rightarrow \underline{2n - 1 + n \lc \log n \rc \text{ bits}} \]

% PROBLEM 2
\newproblem
\underline{The Idea:} Put more frequent symbols at smaller depth

\underline{Greedy approach:} Continually merge least frequent symbols/nodes until you have a full ternary tree encoding all symbols. In this case, take the three lowest frequency symbols, and merge them into one root $R$. Then from the set of the remaining nodes and $R$, take the three lowest nodes and merge them continuously until a full ternary tree is made.

\underline{Optimal Proof:} Suppose we have a tree $T$ with three lowest frequent symbols not as deep as possible. Then at least one has a smaller depth. Switch it with one of the deepest nodes that is more frequent.This improves the encoding length. Thus $T$ is not optimal. Since $T$ is not optimal, then we know that the three lowest frequencies must be at the largest depth. 

\begin{algorithm}
    
    \caption{Ternary Huffman Algorithm}

    \SetKwInOut{KwIn}{Input}
    \SetKwInOut{KwOut}{Output}

    \KwIn{$f = f[1],\cdots,f[n]$; $\Gamma$ has $n$ symbols}
    \KwOut{$T$}

    
    
    $T =$ empty tree;
    
    $H =$ priority queue ordered by $f$;
    
    \For {$i= 1 \text{ to } n$} {
        
        insert($H$, $i$);
    
    }

    \For {$k= (n + 1) \text{ to } (2n - \lc \frac{n}{2} \rc)$} {
        
        $i=$ extract min($H$);
        
        $j=$ extract min($H$);
        
        $k=$ extract min($H$);
        
        Create a node $z$ in $T$ with children $i,j, \text{ and } k$ ;
        
        $f$[$z$]= $f$[$i$] $+$ $f$[$j$] $+$ $f$[$k$];
        
        insert($H$, $z$);
    
    }

    return $T$

\end{algorithm}

% PROBLEM 3
\newproblem
\underline{Proof by contradiction:} Suppose greedy is not optimal solution.

Consider the solution given by the greedy algorithm as a sequence of packages, here represented by indexes: $1, 2, 3, ... n$.
Each package $i$ has a weight, $w_i$, and an assigned truck $t_i$ which is a non-decreasing sequence (as the $k^{th}$ truck is sent out before anything is placed on the $(k+1)^{th}$ truck).

Note that: If $t_n = m$, that means our solution takes $m$ trucks to send out the $n$ packages.

If the greedy solution is non-optimal, then there exists another solution $t'_{i}$, with the same constraints, such that $t'_n = m' < t_n = m$.
Consider the optimal solution that matches the greedy solution as long as possible, so $\forall i < k, t_i = t'_i, \text{ and } t_k \neq t'_k$ (that means, look for the largest $i$ such that: $t_i = t'_i$, and the number of trucks optimal solution takes to send out $k$ packages is not equal to what greedy algorithm is going to take).

It leads to $2$ cases:

\underline{Case $1$:} If $t'_k < t_k$, the greedy solution used more trucks than the optimal solution. As previously mentioned, $\forall$ $i < k, t_i = t'_i$, we know that until $i = (k-1)$, the optimal solution and greedy solution were carrying the same packages, however, the greedy solution used another truck for the $k^{th}$ package. Therefore, we have to add another truck in optimal solution too which leads to the contradiction as we get on the optimal solution with a larger $t'_k$.

\underline{Case $2$:} If $t'_k > t_k$, the optimal solution used more trucks than the greedy solution. We get a contradiction here because greedy solution carried $k$ packages in less number of trucks than the optimal solution which is a contradiction of optimality that greedy solution used less number of trucks than the optimal solution.

(Note that, if $t'_k = t_k$, then greedy solution would be already considered as an optimal solution.)

Therefore, the greedy algorithm is the optimal solution that actually minimizes the number of trucks that are needed.

\end{document} 