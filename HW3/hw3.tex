\documentclass[letterpaper,11pt]{article}

\usepackage{geometry}
\usepackage{pslatex}
\usepackage{fancyhdr}
\usepackage{graphicx}
\usepackage{color}
\usepackage{enumitem} % for ordered list labels
\usepackage{amssymb} % for symbols
\usepackage{scrextend} % for indentation
\usepackage{tabto} % for tabs
\usepackage{amsmath} % for text in equation
\usepackage{listings} % to highlight code

\graphicspath{ {./} }
\geometry{ margin = 1.0in }

%%% TODO modify these variables %%%
\def\homeworknum{3}
\def\myname{Harshit Jain}
\def\myaccessid{hmj5262}
\def\myrecitation{8}
%%%%

\pagestyle{fancy}
\lhead{{\bf CMPSC 465 Fall 2022}}
\chead{{\bf Assignment~\homeworknum}}
\rhead{{\bf \today}}

\newcounter{problemid}
%\stepcounter{problemid}
\def\newproblem{\clearpage\newpage{\bf Problem~\arabic{problemid}\stepcounter{problemid}}\hfill\fbox{\parbox{0.16\textwidth}{\bf Points:}}\par}

\setlength\parindent{0em}
\setlength\parskip{8pt}
\setlength{\fboxsep}{6pt}


\begin{document}

\framebox[\textwidth]{
	\parbox{0.96\textwidth}{
		\parbox{0.12\textwidth}{\bf Name:}\parbox{0.6\textwidth}{\myname}\\
		\parbox{0.12\textwidth}{\bf Access ID:}\parbox{0.6\textwidth}{\myaccessid}\\
		\parbox{0.12\textwidth}{\bf Recitation:}\parbox{0.6\textwidth}{\myrecitation}
	}
}


%% your solutions %%%

\newproblem
\textbf{Acknowledgements}
\begin{enumerate}[label=(\alph*)]
    \item I did not work in a group.
    \item I did not consult without anyone my group members.
    \item I used USTC algorithms book, chapter 9: Sorting in Linear Time as a reference.
\end{enumerate}


% PROBLEM 1
\newproblem
\textbf{Solving recurrences}
\[T(n) = aT(n/b) + \theta (n^d)\]
\begin{enumerate}[label=(\alph*)]
    
    \item Here, $d=1.3$ and $\log_b a = \log_5 11 = 1.48$
    
    So, $\log_b a > d$, we will use case-3 of Master's Theorem:

    $T(n) = \underline{\theta (n^{\log_5 11}) = \theta (n^{1.48})}$
    
    
    \item Here, $d=2.8$ and $\log_b a = \log_2 6 = 2.58$
    
    So, $d > \log_b a$, we will use case-1 of Master's Theorem:

    $T(n) = \underline{\theta (n^{2.8})}$

    
    \item Here, $d=0$ and $\log_b a = \log_3 5 = 1.46$
    
    So, $\log_b a > d$, we will use case-3 of Master's Theorem:

    $T(n) = \underline{\theta (n^{\log_3 5}) = \theta (n^{1.46})}$
    
    
    \item $T(n) = T(n-2) + \log(n)$
    
    \tabto{24pt} $ = [T(n-4) + \log (n-2)] + \log(n)$

    \tabto{24pt} $ = [T(n-6) + \log (n-4)] + \log(n-2) + \log(n)$

    \tabto{80pt} $\vdots$

    \tabto{24pt} $ = [T(n-2k) + \log (n-(2k-2))] + \cdots + \log(n-2) + \log(n)$

    Let $2k = n$,

    \tabto{24pt} $ = T(0) + \log(2) + \log(4) + \log(6) + \cdots + \log(n)$

    \tabto{24pt} $ = 1 + \log(2 \cdot 4 \cdot 6 \cdot \cdots n)$

    \tabto{24pt} $ = 1 + \log(2^{n/2} \cdot (1 \cdot 2 \cdot 3 \cdot \cdots \cdot n/2))$

    \tabto{24pt} $ = 1 + (n/2)(\log(2)) + \log((n/2)!)$

    $\Rightarrow \log(1) + \cdots + \log(1) < \log(1) + \log(2) + \cdots + \log(n/2) < \log(n) + \cdots + \log(n)$
    
    $\Rightarrow 0 < \log((n/2)!) < \log((n/2)^{n/2})$

    Also, $\log((n/2)^{n/2}) = (n/2) \cdot \log(n/2) = \frac{n \cdot \log_c(n)}{2 \cdot \log_c(2)} = O(n \log(n))$

    \tabto{24pt} $ = \underline{O(n \log(n))}$

\end{enumerate}



% PROBLEM 2
\newproblem
\textbf{Sorted Array}

\begin{lstlisting}[language=Python]
    def Search(low, high, A):
    if(low == high):
        if(A[low] == l):
            return low
        else:
            return False
    else:
        mid = (low+high)//2
        if(A[mid] == mid):
            return mid
        elif(A[mid] > mid):
            return Search(low, mid-1, A)
        else:
            return Search(mid+1, high, A)
\end{lstlisting}

    \text{Run-time Analysis}: $\underline{O(\log(n))}$


% PROBLEM 3
\newproblem
\textbf{Linear Time Sorting}

\begin{lstlisting}[language=Python]
    def linearTimeSorting(unsortedList, k):
        sortedList = [0 for _ in range(len(unsortedList))]
        tempList = [0 for _ in range(k+1)]
        
        for j in range(0, len(unsortedList)):
            tempList[unsortedList[j]] += 1

        for i in range(1, k+1):
            tempList[i] = tempList[i] + tempList[i-1]

        for j in range(len(unsortedList)-1, -1, -1):
            sortedList[tempList[unsortedList[j]]-1] = unsortedList[j]
            tempList[unsortedList[j]] -= 1

        return sortedList
\end{lstlisting}

Initially, the $tempList$ is initialized and it takes $O(k)$ time where $k$ is $\underset{i}{max} (x_i)$.

The first `for loop' is making $tempList[i]$ to hold the number of input elements equal to 
$i$ for each integer $i = 1, 2, \cdots ,k \Rightarrow$ frequency of elements. It takes $O(n)$ time. 

The second `for loop' is keeping running sum of array $tempList$ to get how many input elements 
$\leq i$. Tt takes $O(k)$ time.

The third `for loop' placing each element $unsortedList[j]$ in its correct sorted position 
in the $sortedList$; also we decrement $tempList[unsortedList[j]]$ each time we place a value 
$unsortedList[j]$ into the $sortedList$. It takes $O(n)$ time.

Therefore, the time is $O(k+n)$. Note that, there is a significance of $\underset{i}{min} (x_i)$ 
that the $tempList$ starts changing from the index $ = \underset{i}{min} (x_i)$ and not from index $0$ everytime.
This is because the given list may or may not have the minimum integer as $0$. Depending on what minimum number
the list contains, the $tempList$ is starting to change from that index instead of $0$.

Therefore, the \textbf{Overall Time Complexity is:} $O(n + \underset{i}{max} (x_i) - \underset{i}{min} (x_i))$

\tabto{200pt} $\Rightarrow \underline{O(n+M)}$ (where $M = \underset{i}{max} (x_i) - \underset{i}{min} (x_i))$

The $\Omega(n \cdot \log(n))$ does not apply in this case because it is not a comparison sort.
There is no comparison among input elements in this algorithm. 
\end{document} 
