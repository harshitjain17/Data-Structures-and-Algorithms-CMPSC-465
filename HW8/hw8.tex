\documentclass[letterpaper,11pt]{article}

\usepackage{geometry}
\usepackage{pslatex}
\usepackage{fancyhdr}
\usepackage{graphicx}
\usepackage{color}
\usepackage{enumitem} % for ordered list labels
\usepackage{amssymb} % for symbols
\usepackage{scrextend} % for indentation
\usepackage{tabto} % for tabs
\usepackage{amsmath} % for text in equation
\usepackage{forest, tikz} % to make forest
\usepackage[ruled, lined, linesnumbered, commentsnumbered, longend]{algorithm2e}

\graphicspath{ {./} }
\geometry{ margin = 1.0in }

%%% TODO modify these variables %%%
\def\homeworknum{8}
\def\myname{Harshit Jain}
\def\myaccessid{hmj5262}
\def\myrecitation{8}
%%%%

\pagestyle{fancy}
\lhead{{\bf CMPSC 465 Fall 2022}}
\chead{{\bf Assignment~\homeworknum}}
\rhead{{\bf \today}}

\newcounter{problemid}
%\stepcounter{problemid}
\def\newproblem{\clearpage\newpage{\bf Problem~\arabic{problemid}\stepcounter{problemid}}\hfill\fbox{\parbox{0.16\textwidth}{\bf Points:}}\par}

\setlength\parindent{0em}
\setlength\parskip{8pt}
\setlength{\fboxsep}{6pt}


\begin{document}

\framebox[\textwidth]{
	\parbox{0.96\textwidth}{
		\parbox{0.12\textwidth}{\bf Name:}\parbox{0.6\textwidth}{\myname}\\
		\parbox{0.12\textwidth}{\bf Access ID:}\parbox{0.6\textwidth}{\myaccessid}\\
		\parbox{0.12\textwidth}{\bf Recitation:}\parbox{0.6\textwidth}{\myrecitation}
	}
}


%% your solutions %%%

\newproblem
\textbf{Acknowledgements}
\begin{enumerate}[label=(\alph*)]
    \item I worked with Yug Jarodiya.
    \item I did not consult with anyone in my group members.
    \item I did not consult any non-class materials.
\end{enumerate}


% PROBLEM 1
\newproblem
    The minimum cut of a weighted graph is defined as the minimum sum of weights of edges that, when removed from the graph, divide the graph into two sets.
    
    \begin{algorithm}
        \caption{UniqueMinimumCut}

        \SetKwFunction{UniqueMinimumCut}{UniqueMinimumCut}
        \SetKwInOut{KwIn}{Input}
        \SetKwInOut{KwOut}{Output}

        \KwIn{$G = (V,(E, \ell_e))$}
        \KwOut{A unique cut is present or not}

        $C$ = STCut($G$)
        
        $|C|$ = CapacityOfCut($C$)
        
        \For {$e_i \in C$} {

            capacity($e_i$) $+= 1$

            $|C_i| =$ STCut($G$)

            \If{$|C| == |C_i|$ and $C \neq C_i$} {

                return "Min-cut is not unique"

            }

        }

        return "Min-cut is unique"
        
    \end{algorithm}

    Conversely, if there is a different minimum cut $C'$ in the original graph, there will be some $e_i \in C$ that is not in $C'$, so increasing the capacity of that edge will not change the volume of $C'$, thus $|C| = |C_i|$. In conclusion, the graph has a unique minimum cut iff $|C| < |Ci|$ $\forall$ $i$. The algorithm takes at most $n + 1$ computing of minimum cuts, and therefore runs in \textbf{polynomial time}.
    
% PROBLEM 2
\newproblem
\underline{The cut property:} Let A be a subset of edges of some MST of $G = (V,E)$. Let $(S, V-S)$ be a cut that respects $A$. Let $e$ be the lightest edge across the cut. Then $A \cup {e}$ is part of the MST. 

Using the cut property, we can take edge $e$ which is the lightest edge in the subgraph $H$, so we know that edge $e \in T \cap H$ because it is the lightest edge and will preserve among $H$.
We can take edge $e$ as the cut edge for $H$ and so the cut set is $(H \cap S, H-S)$ which is defined in the cut property above. Because the edge $e \in T \cap H$, that same cut exists in an MST of $H$.


% PROBLEM 3
\newproblem
$\Rightarrow$ Given that $G=(V,E)$ and $V' \subset V$.

$\Rightarrow$ Edge $(u,v)$ is present in $T'$, so this the edge is the least weighted edge here.

$\Rightarrow$ Now, the MST has $V'$ edges and contains no cycle. Now, add the edge $(u,v)$ to the MST.

$\Rightarrow$ The cycle includes edge $(u,v)$ and other edges which are part of $T$, you can notice that one cycle forms in $T$ due to the addition of edge $(u,v)$.

$\Rightarrow$ In this cycle, select an edge where weight is highest, it certainly will not be the edge $(u,v)$ because edge $(u,v)$ weights the least.
Hence you will select an edge from your earlier minimum spanning tree $T$. Let the edge be $E'$.

$\Rightarrow$ Remove $E'$ from $T$ and add edge $(u,v)$. Let this tree be called $T'$.

$\Rightarrow$ If we remove one edge from this cycle, the connectivity of new tree will still be maintained.

$\Rightarrow$ As there are $V'$ edges, there are no cycles in $T'$. So the $T'$ is the spanning tree.

$\Rightarrow$ In order to be minimum spanning tree, Weight($T'$) $=$ Weight($T$) $-$ ($E'$ - edge $(u, v)$ ).

$\Rightarrow$ As per the follows, we got to know that ($E'$- edge $(u,v)$) is the positive entity.

$\Rightarrow$ Hence Weight($T'$) $<$ Weight($T$), therefore $T$ cannot be minimum spanning tree. ($T'$) is the minimum spanning tree which has edge($u,v$).

$\Rightarrow$ \textbf{Thus, $T'$ is the minimum spanning tree of $G'$.}

\end{document} 





